\section{Závěr}
Úlohou projektu bylo především si vykoušet techniky přednášené v~předmětech IFJ a IAL prakticky,
naučit se metody výstavby překladačů, získat další znalosti z~programování v~týmu a také vzájemnou spolupráci.

Projekt byl pro nás velkou \textbf{výzvou} a svou odborností převyšoval vše, s~čím jsme do té doby měli dočinění - práce na projektu nám přinesla velké množství zkušeností.

Kromě správnosti jsme se také zaměřovali na \textbf{správnou dekompozici} problémů a snahu o~\uv{čistý} kód. Výsledný překladač je obohacen o~7 rozšíření a byl vyvíjen dle normy \emph{C11}, je tedy nezávislý na platformě. Repozitář tohoto projektu bude po domluvě zveřejněn na serveru GitHub\footnote{Projekt bude po termínu odevzdání zveřejněn na adrese \href{https://github.com/thejoeejoee/IFJ-VUT-BIT-2017-2018/}{https://github.com/thejoeejoee/IFJ-VUT-BIT-2017-2018/}.}.

Z implementace optimalizací nám vyvstanuly některé problémy týkající se kompilace - z těch méně významných jde především o spotřebu paměti - i u jednoduchých programů se jedná o \textbf{desítky MB požadované paměti}. Vyhodnocováním konstatních výrazů jsme také narazili na problém přesnosti předávání operandů v jazyce \ic|IFJcode17| do interpretu tohoto jazyka - řešili jsme v kooperaci s zadávajícími, kteří nám vyhověli a upravili \ic|IFJcode17| pro příjem operandů typu \ic|float| s vyšší přesností.

\emph{
	Z~metrik projektu by bylo vhodné zmínit počet majoritních souborů, 40 .c, 40 .h a 25 .cpp jednotkových testů, celkem \emph{24805} řádků zdrojového kódu a \emph{903} commitů v~systému \emph{Git}, průměrně \emph{10} denně za aktivní dny vývoje.
}

\vfill
\section{Použitá literatura, software a služby}

\renewcommand\labelenumi{[\arabic{enumi}]}
\renewcommand\theenumi\labelenumi
\begin{enumerate}
%literatura
	\item Alexander Meduna a Roman Lukáš, \emph{Formální jazyky a
překladače}, podklady k přednáškám, 17. září 2017

	\item Anurag Singh, \uv{Tarjan’s Algorithm to find Strongly Connected Components}, \emph{GeeksforGeeks}, získáno 22. listopadu 2017 z \href{http://www.geeksforgeeks.org/tarjan-algorithm-find-strongly-connected-components/}{http://www.geeksforgeeks.org/tarjan-algorithm-find-strongly-connected-components/}

	\item \emph{Optimizing compiler}, Wikipedia, získáno 20. listopadu 2017 z \href{https://en.wikipedia.org/wiki/Optimizing\_compiler}{https://en.wikipedia.org/wiki/Optimizing\_compiler}

	\item \emph{One-pass compiler}, Wikipedia, získáno 10. října 2017 z \href{https://en.wikipedia.org/wiki/One-pass\_compiler}{https://en.wikipedia.org/wiki/One-pass\_compiler}

	\item International Organization for Standardization, \emph{C standard ISO/IEC 9899:201x}, 2. prosinec 2010, \\\href{http://www.open-std.org/jtc1/sc22/wg14/www/docs/n1548.pdf}{http://www.open-std.org/jtc1/sc22/wg14/www/docs/n1548.pdf}

	\item cppreference.com, \emph{C and C++ reference}, 1. prosinec 2017, \href{http://en.cppreference.com/}{http://en.cppreference.com/}
\end{enumerate}

\renewcommand\labelenumi{[\Alph{enumi}]}
\renewcommand\theenumi\labelenumi
\vspace{20pt}
\begin{enumerate}
	\item  Josef Kolář a Son Hai Nguyen, \emph{Advánc IFJcode17 IDE}, 1.4.10, 3. prosinec 2017,\\\href{https://github.com/thejoeejoee/VUT-FIT-IFJ-2017-toolkit}{https://github.com/thejoeejoee/VUT-FIT-IFJ-2017-toolkit}
	\item JetBrains s.r.o., CLion, 2017.3, 30. listopad 2017, \href{https://www.jetbrains.com/clion/}{https://www.jetbrains.com/clion/}
	\item The Qt Company, Qt Creator IDE, 4.4.1, 6. říjen 2017, \href{https://www.qt.io/qt-features-libraries-apis-tools-and-ide/\#ide}{https://www.qt.io/qt-features-libraries-apis-tools-and-ide/\#ide}
	\item Microsoft Corporation, Visual Studio Community 2017, 15.4.3, 8. listopad 2017,\\\href{https://www.visualstudio.com/cs/vs/community/}{https://www.visualstudio.com/cs/vs/community/}
	\item GNU Project, \emph{GCC}, 5.4, 3. červen 2016, \href{https://gcc.gnu.org/}{https://gcc.gnu.org/}
	\item MinGW Project, \emph{MinGW}, 5.0.3, 4. listopad 2017, \href{http://mingw.org/}{http://mingw.org/}
	\item Junio Hamano and others, \emph{Git}, 2.15.1, 28. listopad 2017, \href{https://git-scm.com/}{https://git-scm.com/}
\end{enumerate}

\renewcommand\labelenumi{[\Roman{enumi}]}
\renewcommand\theenumi\labelenumi
\vspace{20pt}
\begin{enumerate}
	\item GitHub, Inc, \emph{GitHub}, 2017, \href{https://github.com/}{https://github.com/}
	\item Codecov, \emph{Codecov}, 2017, \href{https://codecov.io/}{https://codecov.io/}
	\item Travis CI, GmbH, \emph{Travis CI}, 2017, \href{https://travis-ci.com/}{https://travis-ci.com/}
\end{enumerate}