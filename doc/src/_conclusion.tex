\section{Závěr}
Úlohou projektu bylo především si vykoušet techniky přednášené v~předmětech IFJ a IAL prakticky,
naučit se metody výstavby překladačů, získat další znalosti z~programování v~týmu a také vzájemnou spolupráci.

Projekt byl pro nás velkou \textbf{výzvou} a svou odborností převyšoval vše, s~čím jsme do té doby měli dočinění - práce na projektu nám přinesla velké množství zkušeností.

Kromě správnosti jsme se také zaměřovali na \textbf{správnou dekompozici} problémů a snahu o~\uv{čistý} kód. Výsledný překladač je obohacen o~7 rozšíření a byl vyvíjen dle normy \emph{C11}, je tedy nezávislý na platformě. Repozitář tohoto projektu bude po domluvě zveřejněn na serveru GitHub.

\emph{
	Z~metrik projektu by bylo vhodné zmínit počet majoritních souborů, 40 .c, 40 .h a 25 .cpp jednotkových testů, celkem \emph{24805} řádků zdrojového kódu a \emph{903} commitů v~systému \emph{Git}, průměrně \emph{10} denně za aktivní dny vývoje.
}