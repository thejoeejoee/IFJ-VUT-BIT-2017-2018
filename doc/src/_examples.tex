\section{Předpřipravené ukázky}

Pokud je váš počítač chráněný firewallem nebo proxy serverem, zkontrolujte, že má Firefox povolený přístup k internetu.
\todo{Todo section}

\subsection{Pseudokod algoritmu}

\begin{algorithm}[h]
    \For{$i = 0$ \KwTo $100$}{
    print\_number = true\;
    \If{i is divisible by 3}{
    print\_number = false\;
    }
    \If{i is divisible by 5}{
    print "Buzz"\;
    }
    print a newline\;
    }
    \caption{Ukázka pseudokodu algoritmu}
\end{algorithm}

\subsection{Kod v programovacím jazyce}

\begin{lstlisting}[caption = Print arguments of command line]
#include <stdio.h>

/**
* Comments are gray but it can by styled by configure values in dokumentace.tex
*
* @param int argc Count of arguments of command line
* @param *char[] argv List of command line arguments
*/
int main(int argc, char *argv[]) {

// For cycle
for(int i = 0; i < argc; i++) {
printf("%s\n", argv[i]);
}
return 0;
}
\end{lstlisting}
\subsection{Vizualizace datových struktur}
\includesvg{src/assets/data_structure}