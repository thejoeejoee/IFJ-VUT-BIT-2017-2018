\subsection{Lexikální analýza}
\label{subsec:lexer}
Úlohou Lexikálního analyzátoru je rozpoznávat jazykové lexémy a reprezentovat
je jako tokeny.

Lexikální analyzátor je implementací deterministického konečného
automatu. Jediný nedeterministický případ nastává tehdy, když má automat načítat znak ve tvaru $\backslash$\emph{ddd}.
Případ je vyřešen implementací interního čítače počtu načtených číslic. Pokud čítač dosáhne hodnoty 3 a načtená hodnota je
v~intervalu $\langle 001 - 255\rangle$, je znak úspěšné načten, v~opačném případě je ohlášena chyba v~lexikální
analýze.

Lexikální analyzátor je rozdělen do tří modulů: \ic|lexer.c|, \ic|lexer_fsm.c| a \ic|token.c|
. Modul \ic|token.c| obsahuje implementaci abstraktního datového typu token. Modul \ic|lexer_fsm.c|
obsahuje implementaci \textbf{deterministického konečného automatu}. Automat je oddělen od samotné implementace řízení lexikální analýzy z~důvodu snažšího testování veškerých přechodových stavů automatu mezi jednotlivými konfiguracemi. Modul \ic|lexer.c| poskytuje funkci
\ic|lexer_next_token|, která řídí činnost deterministického konečného automatu a vrací další token. Chyba v~lexikální analýze je rozpoznána \emph{navrácením tokenu s~příznakem chyby}.

\begin{figure}[htbp]
    \label{subsec:automat}
    \caption{Grafická reprezentace konečného automatu}
    \centering
    \includegraphics[width=0.9\textwidth, angle=0]{src/assets/automat.pdf}
\end{figure}

\subsection{Synataktická analýza}
Syntaxí řízený překlad je naimplementován v~modulu \ic|parser.c|, pro syntaktickou analýzu programu jako takového je
použita SA shoda dolů, konkrétně metoda \emph{rekurzivního sestupu}. Pro každé pravidlo z~LL gramatiky existuje v~tomto
modulu funkce realizující toto pravidlo.

Pro snažší zápis v~jazyce C byl implementován \textbf{poloautomatický systém generování funkcí pravidel} za pomocí maker.
Makra konkrétní pravidlo vykonají a zkontrolují jeho návratovou hodnotu, resp. úspěch kontroly. Pro kontrolu přítomnosti terminálu bylo
naimplemntováno makro \ic|CHECK_TOKEN(TOKEN_TYPE)|, pro spuštění jiného pravidla poté najri
\ic|CALL_RULE(NAME_OF_RULE)|.

Pro podmínečná volání pravidel makro
\ic|CHECK_RULE(type==TOKEN_TYPE,| \ic|NAME_OF_RULE,| \ic|REWIND_AND_SUCCESS)| - při jeho použití se spustí pravidlo
\ic|NAME_OF_RULE| v~případě, že je následující token typu \ic|TOKEN_TYPE| a jestliže bylo úspěšné, další načtený
token navrátí zpět lexikálnímu analyzátoru a prohlásí se volání za úspěšné.

Všechny pravidla pracují nad strukturou \ic|Parser|, která zapouzdřuje základní komponenty pro syntaxí řízený překlad.
Jedná se především od struktury \ic!ParserSemantic!, \ic!CodeConstructor! a \ic!Lexer!. První zmíněná zajištuje sémantické
kontroly programu; uchovává \emph{registr symbolů}, dočasné proměnné, pravidla pro implicitní konverze datových typů či
aktuální scénář SA, o~dalších kapitolách pojednávají jiné kapitoly. Po splnění některých pravidlech se v~SA vyskytují volání konstrukce kódu, kde je využito rozhraní z~modulu \ic!CodeConstructor! - SA tedy \textbf{není přímo zodpovědná za výslednou podobu cílového kódu}.

Některé derivace neterninálu \ic|<statement_single>| nemohou v~určitém kontextu nastat
- např. příkaz \ic|return| v~hlavním bloku programu. Syntaktická analýza si v~tomto případě \uv{vypomáhá} \textbf{sémantickými akcemi}, ačkoliv by tento problém mohla řešit sama. Takto upravená gramatika by se velmi rozrostla, což by snížilo její čitelnost.

Pro analýzu výrazů není analýza shora dolů v~případě jazyka \ic!IFJ17! možná, proto byla
použita standardní metoda zdola nahoru, v~našem případě založená na precedenční
syntaktické analýze - implementovaná v~modulu \ic|parser_expr.c|.

V~případě že je jako neterminál očekáván výraz, je volána funkce \ic|parser_parse_expr| ze zmíněného modulu, která provede \textbf{precedenční syntaktickou analýzu} výrazu a opět, jako ostatní pravidla, navrátí příznak úspěchu. V~rámci zpracování výrazů pomocí PA je také přímo generován cílový kód pomocí rozhraní modulu \ic|CodeGenerator|.

Neterminál \ic|<eols>| slouží ke \uv{zkonzumování} všech přebytečných znaků nového řádku - resp. terminálu \ic!EOL!. Jedná se o~univerzální neterminál použitý pro všechny možné výskyty většího množství znaků konců řádků. Proto je pro všechny terminály (kromě samotného \ic|EOL|) tohoto použito pravidlo \emph{60}.

\subsubsection{Gramatika a LL tabulka}
\begin{normalsize}
    \begin{enumerate}
        {\small
        \item <prog> $\rightarrow$ <body> <eols> EOF
        \item <body> $\rightarrow$ <definitions\_block> <scope> <shared\_variables\_declarations>

        \item <definitions\_block> $\rightarrow$ <eols> <definitions>

        \item <definitions> $\rightarrow$ <definition> <definitions\_block>
        \item <definitions> $\rightarrow$ $\varepsilon$

        \item <definition> $\rightarrow$ <function\_declaration>
        \item <definition> $\rightarrow$ <function\_definition>
        \item <definition> $\rightarrow$ <shared\_variable\_declaration>

        \item <function\_definition> $\rightarrow$ <function\_header> EOL <eols> <statements> END FUNCTION EOL
        \item <function\_declaration> $\rightarrow$ DECLARE <function\_header> EOL

        \item <function\_header> $\rightarrow$ FUNCTION IDENTIFIER (<function\_params>) AS <type> EOL

        \item <function\_params> $\rightarrow$ $\varepsilon$
        \item <function\_params> $\rightarrow$ <function\_param> <function\_n\_param>

        \item <function\_n\_param> $\rightarrow$ $\varepsilon$
        \item <function\_n\_param> $\rightarrow$ , <function\_param> <function\_n\_param>

        \item <function\_param> $\rightarrow$ IDENTIFIER AS <type>


        \item <type> $\rightarrow$ INTEGER
        \item <type> $\rightarrow$ BOOLEAN
        \item <type> $\rightarrow$ STRING
        \item <type> $\rightarrow$ DOUBLE

        \item <statements> $\rightarrow$ $\varepsilon$
        \item <statements> $\rightarrow$ <statement\_single> EOL <eols> <statements>


        \item <statement\_single> $\rightarrow$ <identifier\_assignment>
        \item <statement\_single> $\rightarrow$ <input>
        \item <statement\_single> $\rightarrow$ <return>
        \item <statement\_single> $\rightarrow$ <print>
        \item <statement\_single> $\rightarrow$ <condition>
        \item <statement\_single> $\rightarrow$ <while\_>
        \item <statement\_single> $\rightarrow$ <variable\_declaration>
        \item <statement\_single> $\rightarrow$ <static\_variable\_declaration>
        \item <statement\_single> $\rightarrow$ <scope>

        \item <variable\_declaration> $\rightarrow$ DIM IDENTIFIER AS <type> <declaration\_assignment>
        \item <declaration\_assignment> $\rightarrow$ E
        \item <declaration\_assignment> $\rightarrow$ <assignment>

        \item <shared\_variables\_declarations> $\rightarrow$ E
        \item <shared\_variables\_declarations> $\rightarrow$ <shared\_variable\_declaration>
        \item <shared\_variable\_declaration> $\rightarrow$ DIM SHARED IDENTIFIER AS <type> <declaration\_assignment>

        \item <static\_variable\_declaration> $\rightarrow$ STATIC IDENTIFIER AS <type> <declaration\_assignment>

        \item <return> $\rightarrow$ RETURN <expr>

        \item <assignment> $\rightarrow$ = <expression>
        \item <assignment> $\rightarrow$ <modify> <expression>
        \item <modify> $\rightarrow$ +=
        \item <modify> $\rightarrow$ -=
        \item <modify> $\rightarrow$ *=
        \item <modify> $\rightarrow$ /=
        \item <modify> $\rightarrow$ $\backslash$=

        \item <print> $\rightarrow$ PRINT <print\_expression> <print\_expressions>
        \item <print\_expressions> $\rightarrow$ E
        \item <print\_expressions> $\rightarrow$ <print\_expression> <print\_expressions>
        \item <print\_expression> $\rightarrow$ <expression> SEMICOLON

        \item <while\_> $\rightarrow$ DO WHILE <expression> EOL <eols> <cycle\_statements> LOOP

        \item <input> $\rightarrow$ INPUT IDENTIFIER

        \item <identifier\_assignment> $\rightarrow$  IDENTIF <assignment>

        \item <condition> $\rightarrow$ IF <expr> THEN EOL <eols> <statements> <condition\_elseif> <condition\_else> END IF
        \item <condition\_elseif> $\rightarrow$ $\varepsilon$
        \item <condition\_elseif> $\rightarrow$ ELSEIF <expr> THEN EOL <eols> <statements> <condition\_elseif>

        \item <condition\_else> $\rightarrow$ $\varepsilon$
        \item <condition\_else> $\rightarrow$ ELSE EOL <eols> <statements>

        \item <eols> $\rightarrow$ $\varepsilon$
        \item <eols> $\rightarrow$ EOL <eols>
        }
        \newpage
        \begin{landscape}
            \begin{table}[htbp]
                \label{table:prec}
                \centering
                \caption{LL tabulka 1. část}
                \begin{tabular}{|l|l|l|l|l|l|l|l|l|l|l|l|l|l|l|l|l|l|l|l|l|l|l|l|l|}
                    \hline
                    & {\rotatebox[origin=c]{90}{Operátor}}  & ( & ) & {\rotatebox[origin=c]{90}{identifier}}
                    & {\rotatebox[origin=c]{90}{integer literal}} & = & ; & {\rotatebox[origin=c]{90}{as}}
                    & {\rotatebox[origin=c]{90}{asc}}

                    & {\rotatebox[origin=c]{90}{delcare}} & {\rotatebox[origin=c]{90}{dim}}
                    & {\rotatebox[origin=c]{90}{do}} & {\rotatebox[origin=c]{90}{double}}
                    & {\rotatebox[origin=c]{90}{else}} & {\rotatebox[origin=c]{90}{end}}
                    & {\rotatebox[origin=c]{90}{chr}} & {\rotatebox[origin=c]{90}{function}}

                    & {\rotatebox[origin=c]{90}{if}} & {\rotatebox[origin=c]{90}{input}}
                    & {\rotatebox[origin=c]{90}{integer}} & {\rotatebox[origin=c]{90}{length}}
                    & {\rotatebox[origin=c]{90}{loop}} & {\rotatebox[origin=c]{90}{print}}
                    & {\rotatebox[origin=c]{90}{return}}
                    \\ \hline

                    <prog>&&&&&&&&&&1&&&&&&&1&&&&&&&
                    \\ \hline
                    <body>&&&&&&&&&&2&&&&&&&2&&&&&&&
                    \\ \hline
                    <def\_b>&&&&&&&&&&3&&&&&&&3&&&&&&&
                    \\ \hline
                    <defs>&&&&&&&&&&4&&&&&&&4&&&&&&&
                    \\ \hline
                    <def>&&&&&&&&&&6&&&&&&&7&&&&&&&
                    \\ \hline
                    <f\_def>&&&&&&&&&&&&&&&&&9&&&&&&&
                    \\ \hline
                    <f\_he>&&&&&&&&&&&&&&&&&10&&&&&&&
                    \\ \hline
                    <f\_pas>&&&12&13&&&&&&&&&&&&&&&&&&&&
                    \\ \hline
                    <f\_n\_p>&&&14&15&&&&&&&&&&&&&&&&&&&&
                    \\ \hline
                    <f\_pa>&&&&16&&&&&&&&&&&&&&&&&&&&
                    \\ \hline
                    <type>&&&&&&&&&&&&&20&&&&&&&17&&&&
                    \\ \hline
                    <sts>&&&&&&&&&&&&&&&21&&&&&&&&&
                    \\ \hline
                    <stsi>&&&23&&&&&&&&29&28&&&&&&27&24&&&&26&25
                    \\ \hline
                    <va\_de>&&&&&&&&&&&32&&&&&&&&&&&&&
                    \\ \hline
                    <va\_as>&&&&&&34&&&&&32&&&&&&&&&&&&&
                    \\ \hline
                    <s\_v\_ds>&&&&&&&&&&35&36&&&&&&35&&&&&&&
                    \\ \hline
                    <st\_v\_d>&&&&&&&&&&35&36&&&&&&35&&&&&&&
                    \\ \hline
                    <ret>&&&&&&&&&&&&&&&&&&&&&&&&39
                    \\ \hline
                    <ass>&&&&&&40&&&&&&&&&&&&&&&&&&
                    \\ \hline
                    <mod>&&&&&&&&&&&&&&&&&&&&&&&&
                    \\ \hline
                    <pr>&&&&&&&&&&&&&&&&&&&&&&&47&
                    \\ \hline
                    <pr\_es>&49&49&49&&&&&&&&&&&&&&&&&&&&48&
                    \\ \hline
                    <pr\_e>&50&50&50&&&&&&&&&&&&&&&&&&&&&
                    \\ \hline
                    <while>&&&&&&&&&&&&51&&&&&&&&&&&&
                    \\ \hline
                    <input>&&&&&&&&&&&&&&&&&&&52&&&&&
                    \\ \hline
                    <id\_a>&&&&53&&&&&&&&&&&&&&&&&&&&
                    \\ \hline
                    <con>&&&&&&&&&&&&&&&&&&54&&&&&&
                    \\ \hline
                    <con\_ei>&&&&&&&&&&&&&&55&55&&&54&&&&&&
                    \\ \hline
                    <con\_e>&&&&&&&&&&&&&&58&57&&&&&&&&&
                    \\ \hline
                    <con\_eols>&59&59&59&59&59&59&59&59&59&59&59&59&59&59&59&59&59&59&59&59&59&59&59&59
                    \\ \hline
                \end{tabular}
            \end{table}
        \end{landscape}
        \newpage
        \begin{landscape}
            \begin{table}[htbp]
                \label{table:prec2}
                \centering
                \caption{LL tabulka 2. část}
                \begin{tabular}{|l|l|l|l|l|l|l|l|l|l|l|l|l|l|l|l|l|l|l|l|l|l|l|l|l|l|l|l|l|l|}
                    \hline


                    & {\rotatebox[origin=c]{90}{scope}}
                    & {\rotatebox[origin=c]{90}{string}} & {\rotatebox[origin=c]{90}{substr}}
                    & {\rotatebox[origin=c]{90}{then}} & {\rotatebox[origin=c]{90}{while}}
                    & {\rotatebox[origin=c]{90}{and}} & {\rotatebox[origin=c]{90}{boolean}}
                    & {\rotatebox[origin=c]{90}{continue}} & {\rotatebox[origin=c]{90}{elseif}}
                    & {\rotatebox[origin=c]{90}{exit}} & {\rotatebox[origin=c]{90}{false}}
                    & {\rotatebox[origin=c]{90}{for}} & {\rotatebox[origin=c]{90}{next}}
                    & {\rotatebox[origin=c]{90}{not}} & {\rotatebox[origin=c]{90}{or}}
                    & {\rotatebox[origin=c]{90}{shared}} & {\rotatebox[origin=c]{90}{static}}
                    & {\rotatebox[origin=c]{90}{true}} & {\rotatebox[origin=c]{90}{double literal}}
                    & {\rotatebox[origin=c]{90}{string value}} & {\rotatebox[origin=c]{90}{comma}}
                    & {\rotatebox[origin=c]{90}{EOL}}
                    & {\rotatebox[origin=c]{90}{error}} & {\rotatebox[origin=c]{90}{EOF}}
                    & {\rotatebox[origin=c]{90}{+=}} & {\rotatebox[origin=c]{90}{-=}}
                    & {\rotatebox[origin=c]{90}{*=}} & {\rotatebox[origin=c]{90}{/=}}
                    & {\rotatebox[origin=c]{90}{\textbackslash=}}
                    \\ \hline
                    <prog>&1&&&&&&&&&&&&&&&1&&&&&&1&&&&&&&
                    \\ \hline
                    <body>&2&&&&&&&&&&&&&&&2&&&&&&2&&&&&&&
                    \\ \hline
                    <def\_b>&3&&&&&&&&&&&&&&&3&&&&&&3&&&&&&&
                    \\ \hline
                    <defs>&5&&&&&&&&&&&&&&&4&&&&&&&&&&&&&
                    \\ \hline
                    <def>&5&&&&&&&&&&&&&&&8&&&&&&&&&&&&&
                    \\ \hline
                    <f\_def>&&&&&&&&&&&&&&&&&&&&&&&&&&&&&
                    \\ \hline
                    <f\_he>&&&&&&&&&&&&&&&&&&&&&&&&&&&&&
                    \\ \hline
                    <f\_pas>&&&&&&&&&&&&&&&&&&&&&&&&&&&&&
                    \\ \hline
                    <f\_n\_p>&&&&&&&&&&&&&&&&&&&&&&&&&&&&&
                    \\ \hline
                    <f\_pa>&&&&&&&&&&&&&&&&&&&&&&&&&&&&&
                    \\ \hline
                    <type>&&19&&&&&18&&&&&&&&&&&&&&&&&&&&&&
                    \\ \hline
                    <sts>&&&&&&&&&&&&&&&&&&&&&&&&&&&&&
                    \\ \hline
                    <va\_de>&&&&&&&&&&&&&&&&&&&&&&&&&&&&&
                    \\ \hline
                    <va\_as>&&&&&&&&&&&&&&&&&&&&&&33&&&&&&&
                    \\ \hline
                    <s\_v\_ds>&1&&&&&&&&&&&&&&&&&&&&&&&&&&&&
                    \\ \hline
                    <st\_v\_d>&&&&&&&&&&&&&&&&&38&&&&&&&&&&&&
                    \\ \hline
                    <ret>&&&&&&&&&&&&&&&&&&&&&&&&&&&&&
                    \\ \hline
                    <ass>&&&&&&&&&&&&&&&&&&&&&&&&&41&41&41&41&41
                    \\ \hline
                    <mod>&&&&&&&&&&&&&&&&&&&&&&&&&42&43&44&45&46
                    \\ \hline
                    <pr>&&&&&&&&&&&&&&&&&&&&&&&&&&&&&
                    \\ \hline
                    <pr\_es>&&&49&&&&&&&&&&&&&&&&&&&48&&&&&&&
                    \\ \hline
                    <pr\_e>&&&50&&&&&&&&&&&&&&&&&&&&&&&&&&
                    \\ \hline
                    <while>&&&&&&&&&&&&&&&&&&&&&&&&&&&&&
                    \\ \hline
                    <input>&&&&&&&&&&&&&&&&&&&&&&&&&&&&&
                    \\ \hline
                    <id\_a>&&&&&&&&&&&&&&&&&&&&&&&&&&&&&
                    \\ \hline
                    <con>&&&&&&&&&&&&&&&&&&&&&&&&&&&&&
                    \\ \hline
                    <con\_ei>&&&&&&&&&56&&&&&&&&&&&&&&&&&&&&
                    \\ \hline
                    <con\_e>&&&&&&&&&&&&&&&&&&&&&&&&&&&&&
                    \\ \hline
                    <con\_eols>&59&59&59&59&59&59&59&59&59&59&59&59&59&59&59&59&59&59&59&59&59&60&59&59&59&59&59&59&59

                    \\ \hline
                \end{tabular}

            \end{table}
        \end{landscape}
        \newpage
    \end{enumerate}
\end{normalsize}
\subsubsection{Precedenční analýza výrazů}

Precedenční analýza je řízena precedenční tabulkou, kterou se vyhodnocuje pořadí zpracování tokenů.
. V~tabulce se nachází jak \textbf{binární, tak i unární mínus}. Lexikální analyzátor nám ovšem poskytuje pouze jeden token mínus a proto se unární a binární mínus musí vyhodnotit \textbf{podle kontextu}.

Precedenční analýza využívá \textbf{zásobník v~podobě obousměrně vázaného seznamu}, do kterého se ukládají terminály,precedenční symboly a neterminály. Pomocí redukčních pravidel, která jsou vypsána v~příloze, je výraz postupně redukován.
Vzhledem k~tomu, že je překladač založen na přímém překladu, je při redukování pomocí pravidel přímo voláno generování kódu, resp. v~případech konstatních literálů je \textbf{výraz rovnou vyhodnocován}, viz kapitola \ref{subsec:optimization-cee}.

\begin{table}[!htbp]
    \centering
    \label{tabul:prav}
    \caption{Redukční pravidla}
    \begin{tabular}{lll}
        $E \to i$ & $E \to E - E$ & $E \to E >= E$\\
        $E \to (E)$ & $E \to E ~ / ~ E$ & $E \to E < E$\\
        $E \to i()$ & $E \to E ~ \backslash ~ E$ & $E \to E <= E$\\
        $E \to i(E)$ & $E \to - E$ & $E \to NOT ~ E$\\
        $E \to i(E, E)$ & $E \to E = E$ & $E \to E ~ AND ~ E$\\
        $E \to i(E, E, ...)$ & $E \to E <> E$ & $E \to E ~ OR ~ E$\\
        $E \to E + E$ & $E \to E > E$\\
    \end{tabular}
\end{table}

\begin{table}[htbp]
\label{tabul:prec}
\centering
\caption{Precedenční tabulka}
\label{precedencni-tabulka}
\begin{tabular}{|r|c|c|c|c|c|c|c|c|c|c|c|c|c|c|c|c|c|c|c|c|}
\hline
& $un -$ & $*$ & $/$ & $\backslash$ & $+$ & $-$ & $=$ & $<>$ & $<$ & $<=$ & $>=$ & $>$ & $NOT$ & $AND$ & $OR$ & $($ & $)$ & $,$ & $i$ & \$ \\ \hline
$un -$ &$<$&$>$&$>$&$>$&$>$&$>$&$>$&$>$&$>$&$>$&$>$&$>$& x &$>$&$>$&$<$&$>$&$>$&$<$&$>$\\ \hline
$*$ &$<$&$>$&$>$&$>$&$>$&$>$&$>$&$>$&$>$&$>$&$>$&$>$&$<$&$>$&$>$&$<$&$>$&$>$&$<$&$>$\\ \hline
$/$ &$<$&$>$&$>$&$>$&$>$&$>$&$>$&$>$&$>$&$>$&$>$&$>$&$<$&$>$&$>$&$<$&$>$&$>$&$<$&$>$\\ \hline
$\backslash$ &$<$&$<$&$<$&$>$&$>$&$>$&$>$&$>$&$>$&$>$&$>$&$>$&$<$&$>$&$>$&$<$&$>$&$>$&$<$&$>$\\ \hline
$+$ &$<$&$<$&$<$&$<$&$>$&$>$&$>$&$>$&$>$&$>$&$>$&$>$&$<$&$>$&$>$&$<$&$>$&$>$&$<$&$>$\\ \hline
$-$ &$<$&$<$&$<$&$<$&$>$&$>$&$>$&$>$&$>$&$>$&$>$&$>$&$<$&$>$&$>$&$<$&$>$&$>$&$<$&$>$\\ \hline
$=$ &$<$&$<$&$<$&$<$&$<$&$<$& x & x & x & x & x & x &$<$&$>$&$>$&$<$&$>$&$>$&$<$&$>$\\ \hline
$<>$ &$<$&$<$&$<$&$<$&$<$&$<$& x & x & x & x & x & x &$<$&$>$&$>$&$<$&$>$&$>$&$<$&$>$\\ \hline
$<$ &$<$&$<$&$<$&$<$&$<$&$<$& x & x & x & x & x & x &$<$&$>$&$>$&$<$&$>$&$>$&$<$&$>$\\ \hline
$<=$ &$<$&$<$&$<$&$<$&$<$&$<$& x & x & x & x & x & x &$<$&$>$&$>$&$<$&$>$&$>$&$<$&$>$\\ \hline
$>=$ &$<$&$<$&$<$&$<$&$<$&$<$& x & x & x & x & x & x &$<$&$>$&$>$&$<$&$>$&$>$&$<$&$>$\\ \hline
$>$ &$<$&$<$&$<$&$<$&$<$&$<$& x & x & x & x & x & x &$<$&$>$&$>$&$<$&$>$&$>$&$<$&$>$\\ \hline
$NOT$ & x &$>$&$>$&$>$&$>$&$>$&$>$&$>$&$>$&$>$&$>$&$>$&$<$&$>$&$>$&$<$&$>$&$>$&$<$&$>$\\ \hline
$AND$ &$<$&$<$&$<$&$<$&$<$&$<$&$<$&$<$&$<$&$<$&$<$&$<$&$<$&$>$&$>$&$<$&$>$&$>$&$<$&$>$\\ \hline
$OR$ &$<$&$<$&$<$&$<$&$<$&$<$&$<$&$<$&$<$&$<$&$<$&$<$&$<$&$<$&$>$&$<$&$>$&$>$&$<$&$>$\\ \hline
$($ &$<$&$<$&$<$&$<$&$<$&$<$&$<$&$<$&$<$&$<$&$<$&$<$&$<$&$<$&$<$&$<$& = & = &$<$& x \\ \hline
$)$ &$>$&$>$&$>$&$>$&$>$&$>$&$>$&$>$&$>$&$>$&$>$&$>$&$>$&$>$&$>$& x &$>$&$>$& x &$>$\\ \hline
$,$ &$<$&$<$&$<$&$<$&$<$&$<$&$<$&$<$&$<$&$<$&$<$&$<$&$<$&$<$&$<$&$<$& = & = &$<$& x \\ \hline
$i$ & x &$>$&$>$&$>$&$>$&$>$&$>$&$>$&$>$&$>$&$>$&$>$& x &$>$&$>$& = &$>$&$>$& x &$>$\\ \hline
\$ &$<$&$<$&$<$&$<$&$<$&$<$&$<$&$<$&$<$&$<$&$<$&$<$&$<$&$<$&$<$&$<$& x & x &$<$& x\\ \hline
\end{tabular}
\end{table}


\subsection{Sémantická analýza}
Sémantické kontroly jsou přidruženy k~syntaktickým pravidlům v~rekurzivním sestupu a
precedenční syntaktické analýze výrazů. Jsou organizovány do tzv. \textbf{scénářů}. Jeden scénář
sémanticky popisuje související část kódu. - např. definici
či deklaraci funkce. Sémantická analýza nám také pomáhá hlídat i některé \textbf{gramatické aspekty},
které by byly pro LL gramatiku příliš kompikované, nebo by měly příliš dlouhý zápis.

\subsection{Konstruktor kódu}
\label{subsec:code-constructor}
Konstruktor cílového kódu je komponenta překladače, který hlídá význam a odpovědnosti vygenerovaného tříadresného kódu \ic!IFJcode17!. Drží informace o~\textbf{vygenerovaných návěštích pro podmínky, cykly i funkce}, příznak generování kódu do bloku cyklu/vně, registr instrukcí pro konverze datových typů a další.

\subsection{Generátor cílového kódu}
Generátor kódu je nízkoúrovňová komponenta zastřešující skládání, validaci a vykreslování cílového kódu \ic|IFJcode17|.
Kontroluje generované instrukce a její operandy, tedy \textbf{validní kombinace typů operandů} (přístupy do rámců, konstatní
literály, návěští či datové typy) u~konkrétních instrukcí.

Interní implementace spoléhá na \emph{obousměrně svázaný lineární seznam} struktur \ic|CodeInstruction|, které kromě
režijních ukazatelů uchovávají typ instrukce a ukazatele na až tři operandy, struktury \ic|CodeInstructionOperand|.
Tento seznam je uložen v~datové struktuře \ic|CodeGenerator|, která dále obsahuje pole podpisů\footnote{Podpis
instrukce je složena z~bitových masek definující povolené typy operandů a její textové reprezentace v~kódu} \ic!IFJcode17!. instrukcí pro jejich validace.
Struktura \ic|CodeInstructionOperand| uchovává informace o~svém typu a poté unii dat pro konkrétní typ operandu, tedy
ukazatel na proměnnou \ic|SymbolVariable|,
data konstanty \ic|CodeInstructionOperandConstantData| v~unii s~datovým typem nebo řetězec uchovávající název návěští.

\subsection{Optimalizátor kódu}
Následující blok obsahuje výčet implementovaných optimalizačních procesů - ty jsou spoušteny v~našimi experimenty zjištěném pořadí a některé i mnohonásobně s~různou konfigurací - vždy tak, aby nedošlo k~funkčně neekvivalentním úpravám programu. 

\begin{description}
    \item[Značkování hotového cílového kódu] Protože jsme nebyli schopni jednoznačně určit, zda daná instrukce patří do těla funkce, či je součásti výrazu, byla zvolena metoda označení instrukcí již při \emph{syntaktické analýze}, kde je možno určit instrukce, které se vygenerovaly v~rámci určitého pravidla např. výrazu.
    
    \item[Vyhodnocování konstatních výrazů v~PA]\label{subsec:optimization-cee} Optimalizace pracuje již při redukci výrazu pomocí \emph{precedenční syntaktické analýzy}, v~jednotlivých pravidlech se zjišťuje, zda jsou dané \textbf{výrazy konstantní literály} a pokud ano, tak se místo generování diskrétních instrukcí pro vyhodnocení generuje instrukce pro přímé vložení již vyhodnocené hodnoty na základě konstatních literálů. Tato optimalizace tedy jako jediná \textbf{nepracuje s~instrukcemi} jazyka \ic|IFJcode17|.
    
    \item[Odstranění nepoužitých symbolů] Před odstraněním nepoužitých symbolů, se nejdříve \textbf{provede analýza} nagenerovaného \ic|IFJcode17| kódu, kde se vytvoří \emph{tabulka s~rozptýlenými položkami}, kde klíč je unikátní identifikátor symbolu a hodnota je počet použití z~daného symbolu. Avšak přestože je \ic|READ| instrukce zápisu, má věnjší vliv, tudíž inkrementujeme počet použití symbolu. Ostatní instrukce zápisu včetně \ic|DEFVAR| se nepoučítají jako použití proměnné. Následně jsou odstraněny všechny nepoužité funkce a proměnné včetně jejich výrazy do nich zapisující.
    
    \item[Optimalizace \uv{peephole}] Tato optimalizace pracuje nad malými segmenty vygenerovaného kódu. Pracuje tak, že nalezne sekvenci instrukcí, které odpovídají určitému \textbf{vzoru} a následně danou sekvenci \textbf{nahradí efektivnější} (ce se týče ceny \ic|IFJcode17|) sekvencí instrukcí. Tuto optimalizaci jsme navíc rozšířili o~podmíněnost počtu výskytů symbolů, které se mohou objevit v~operandech. Bohužel tato optimalizace může odstranit značky instrukcí, v~prvním běhu jsou použita \uv{mírnější pravidla}, narozdíl od toho druhého, který zároveň porušuje značky.
    
    \item[Interpretace celých konstantních výrazů] Při aktualizaci statistikách o~instrukcích se také zjistí, zda výrazy obsahují pouze literálové operandy a až jednu pomocnou proměnnou, která je používána pro přetypování hodnoty pod vrcholem zásobníku. Jestliže je obsah intepretovatelný, je poté provedena \emph{interpretace}, kterou zapouzdřuje struktura \ic|Interpreter|. Tento interpreter interpretuje \textbf{pouze zásobníkové instrukce}. Toto omezení jsme si mohli dovolit z~důvodu, že všechny výrazy jsou generovány pouze pomocí zásobníkových instrukcí.
    
    \item[Částečná interpretace] Narozdíl od \emph{interpretace celých konstantních výrazů} tato optimalizace pracuje s~co nejmenší možnou interpretovatelnou \emph{sekvencí instrukcí}. Proto tato optimalizace nepracuje s~příznaky instrukcí, ale pouze hledá předem zadanou sekvencí instrukcí, kterou je schopna interpretovat a nahradit na vložení instrukce s~vypočtenou hodnotou.
    
    \item[Šíření konstant] Algoritmus této optimalizace byl ze všech výše zmíněných nejkomplikovanější. Nejdříve je nutno jednotlivé instrukce rozdělit do bloků, které jsou uzly \emph{orientovaného grafu}. Následně se zjistí \emph{silně propojené komponenty}, které označují cykly, či rekurzi a naleznou se všechny proměnné, které byly v~daných blocích modifikovány, tyto proměnné budou vyřazeny při šíření konstant v~daných blocích. Při šíření konstant je použit \emph{zásobník tabulek konstant}. \\Blok, který je spojen s~předchozím blokem podmíněnou hranou např. podmínka, je použita nejvyšší tabulka na zásobníku, která je označená příznakem, že daná tabulka byla vytvořena v~předchozím bloku a do daného bloku vedla přímá hrana. Jestliže je do bloku lze dostat přímým skokem, je použita tabulka z~vrcholu zásobníku a daná tabulka se označí příznakem. \\Po zpracování bloku a všech bloků, do kterých vedla z~daného bloku hrana, je \textbf{vyjmuta} z~vrcholu zásobníku tabulka konstant, která odpovídá tabulce, která byla vytvořena v~daném bloku. Kvůli možnosti funkcí ovlivňovat globální proměnné, jsou \textbf{funkce odříznuté od uzlů} hlavního programu. Následně se začíná šíření konstant ve funkcích s~prázdnými tabulkami, čímž se docílí izolovaného šíření konstant ve funkcích.
\end{description}

\newpage
\section{Rozšíření}
\begin{multicols}{2}

\subsection{SCOPE}
Pro implementaci tohoto rozšíření byla upravena tabulka symbolů na \textbf{zásobníkovou strukturu tabulek} tak, aby byla schopna reagovat vložením/výjmutím na základě průchodu zdrojovým programem.
Takto je možno dosáhnout až potenciálně nekonečné hloubky \textbf{zanoření bloků bez konfliktních symbolů proměnných} - tuto strukturu jsme pracovně nazvali \ic!SymbolRegister! - rozhraní této struktury poté volají konkrétní sémantické akce.

Definici proměnné v~bloku cyklu je poté problém pro konstruktor kódu, který příšlušnou instrukci definující nový symbol umístí před tělo cyklu.

\subsection{GLOBAL}
Rozšíření GLOBAL bylo implementováno v~gramatice přidáním
nových \textbf{pravidel pro definici statických a globálních proměnných}.
Dále byla upraven registr symbolů (jak je popsáno pro \ic|SCOPE|),
ve kterém je nejnižší tabulka vyhrazena
pro globální proměnné. Názvy statických proměnných jsou zaprefixovány
jménem funkce a jsou uloženy v~globálním rámci cílového jazyka. 
\emph{Způsobem implementace by si kompilátor mohl dovolit i možnost inicializace statické proměné ve funkci \textbf{nestaticky} (např. parametrem), ale dle chování FREEBASIC bylo toto potlačeno.}

\subsection{UNARY}
Pro rozšíření UNARY byla implementována nová \textbf{gramatická pravidla} pro zkrácené přířazování - i sémantické akce odpovídají přiřazení, liší se pouze vygenerovaný kód. Unární mínus poté řeší \textbf{PA pomocí speciálního případu} při zpracování.

\subsection{BASE}
Rozšíření bylo implementuje LA - ta je schopna \textbf{rozpoznat
lexémy označující čísla ve dvojkové, osmičkové a šestnáctkové} soustavě a
výsledek převést na desítkové číslo, se kterým již nakládáno dále identicky jako původně s~desítkovým.

\subsection{FUNEXP}
Volání funkcí ve výrazech bylo implementováno na úrovni \textbf{precedenční syntaktické analýzy},
precedenční tabulka obsahuje redukční pravidla pro redukování funkcí, čímž implementuje i
možnost \textbf{libovolného volání funkcí} ve výrazech.

\subsection{IFTHEN}
Implementováno na gramatické úrovni přidáním \textbf{nepovinných částí}
\ic!elseif! a odstraněních povinnosti mít část \ic!else!.
Konstruktor kódu poté používá \textbf{zásobník návěští} pro skoky mezi jednotlivými bloky podmínky.

\subsection{BOOLOP}
Precedenční tabulka byla modifikována, aby podporovala i logické operace \ic|AND|, \ic|OR| a \ic|NOT|.
Těmto operacím přísluší i sémantické akce \textbf{kontrolující povolené datové typy} a konstruování kódu (které je velmi obdobné s~kódem pro matematické operace).

\end{multicols}